\documentclass[11pt,a4paper]{article}
\usepackage[utf8]{inputenc}
\usepackage[spanish]{babel}
\usepackage{graphicx}
\usepackage{geometry}
\usepackage{amsmath}
\usepackage{listings}
\usepackage{xcolor}
\usepackage{tikz}
\usepackage{hyperref}
\usepackage{fancyhdr}
\usepackage{titlesec}
\usepackage{enumitem}
\usepackage{float}
\usepackage{booktabs}

% Configuración de página
\geometry{
    left=2.5cm,
    right=2.5cm,
    top=3cm,
    bottom=3cm,
    headheight=15pt
}

% Colores personalizados
\definecolor{azuloscuro}{RGB}{30,58,138}
\definecolor{azulmedio}{RGB}{59,130,246}
\definecolor{azulclaro}{RGB}{147,197,253}
\definecolor{verde}{RGB}{16,185,129}
\definecolor{rojo}{RGB}{239,68,68}
\definecolor{grisclaro}{RGB}{241,245,249}
\definecolor{grisoscuro}{RGB}{51,65,85}

% Configuración de código
\lstset{
    language=Python,
    basicstyle=\ttfamily\small,
    keywordstyle=\color{azulmedio}\bfseries,
    commentstyle=\color{grisoscuro}\itshape,
    stringstyle=\color{verde},
    numbers=left,
    numberstyle=\tiny\color{grisoscuro},
    stepnumber=1,
    numbersep=5pt,
    backgroundcolor=\color{grisclaro},
    showspaces=false,
    showstringspaces=false,
    showtabs=false,
    frame=single,
    rulecolor=\color{azulclaro},
    tabsize=2,
    captionpos=b,
    breaklines=true,
    breakatwhitespace=false,
    escapeinside={\%*}{*)}
}

% Encabezado y pie de página
\pagestyle{fancy}
\fancyhf{}
\fancyhead[L]{\textbf{Sistema TSP - Arquitectura MVC}}
\fancyhead[R]{\today}
\fancyfoot[C]{\thepage}
\renewcommand{\headrulewidth}{2pt}
\renewcommand{\headrule}{\hbox to\headwidth{\color{azuloscuro}\leaders\hrule height \headrulewidth\hfill}}

% Formato de títulos
\titleformat{\section}
{\Large\bfseries\color{azuloscuro}}
{}
{0em}
{}[\titlerule[2pt]]

\titleformat{\subsection}
{\large\bfseries\color{azulmedio}}
{}
{0em}
{}

% Portada
\title{
    \vspace{2cm}
    \Huge\textbf{Documentación Técnica}\\
    \Large\textbf{Sistema de Algoritmo Genético TSP}\\
    \Large\textbf{Cluster Beowulf - Arquitectura MVC}\\
    \vspace{1cm}
}
\author{
    \large Sistema de Computación Distribuida\\
    \large Arquitectura Modelo-Vista-Controlador
}
\date{\vspace{1cm}\today}

\begin{document}

\maketitle
\newpage
\tableofcontents
\newpage

\section{Introducción}

Este documento describe la arquitectura y el diseño del Sistema de Algoritmo Genético para resolver el problema del Viajero de Comercio (TSP) distribuido en un Cluster Beowulf usando MPI.

\subsection{Objetivo}

Implementar un sistema limpio, modular y escalable utilizando el patrón de diseño Modelo-Vista-Controlador (MVC) para resolver el TSP mediante algoritmos genéticos distribuidos.

\subsection{Características Principales}

\begin{itemize}
    \item Arquitectura MVC bien definida y separada
    \item Soporte para computación distribuida con MPI
    \item Interfaz gráfica simple y funcional
    \item Código modular y fácil de mantener
    \item Documentación completa
\end{itemize}

\section{Arquitectura del Sistema}

\subsection{Patrón Modelo-Vista-Controlador (MVC)}

El sistema está organizado siguiendo el patrón MVC que separa las responsabilidades en tres capas principales:

\begin{itemize}
    \item \textbf{Modelo (Models):} Contiene la lógica de negocio y el acceso a datos
    \item \textbf{Vista (Views):} Maneja la interfaz de usuario y presentación
    \item \textbf{Controlador (Controllers):} Coordina la comunicación entre modelo y vista
\end{itemize}

\begin{figure}[H]
\centering
\vspace{0.5cm}
\begin{tikzpicture}[
    box/.style={rectangle, rounded corners, fill=azulclaro!50, draw=azuloscuro, thick, minimum width=3.5cm, minimum height=2.2cm, text centered, font=\small},
    arrow/.style={->, >=stealth, thick, color=azuloscuro, line width=1.5pt}
]
    \node[box, fill=azulmedio!30] (controller) {Controlador\\\textit{Coordina}};
    \node[box, fill=verde!30, below left=2.5cm and 2.5cm of controller] (model) {Modelo\\\textit{Lógica de negocio}};
    \node[box, fill=azulclaro!30, below right=2.5cm and 2.5cm of controller] (view) {Vista\\\textit{Interfaz}};
    
    \draw[arrow] (controller) -- node[left, font=\footnotesize] {Usa} (model);
    \draw[arrow] (controller) -- node[right, font=\footnotesize] {Actualiza} (view);
    \draw[arrow] (view) -- node[below, font=\footnotesize] {Eventos} (controller);
    \draw[arrow, dashed] (model) -- node[below, font=\footnotesize] {Notifica} (controller);
\end{tikzpicture}
\vspace{0.3cm}
\caption{Arquitectura MVC del Sistema}
\label{fig:mvc_architecture}
\end{figure}

\subsection{Estructura de Directorios}

La estructura del proyecto está organizada de la siguiente manera:

\begin{lstlisting}[caption=Estructura del Proyecto, language=bash]
bueno/
├── models/              # Capa de Modelo
│   ├── __init__.py
│   ├── genetic_algorithm.py
│   ├── mpi_handler.py
│   └── database.py
├── views/               # Capa de Vista
│   ├── __init__.py
│   └── gui.py
├── controllers/         # Capa de Controlador
│   ├── __init__.py
│   └── app_controller.py
├── utils/               # Utilidades
│   ├── __init__.py
│   └── matrix_loader.py
├── config/              # Configuración
│   └── config.py
├── data/                # Datos
│   └── distancias.json
├── main.py              # Punto de entrada
├── requirements.txt     # Dependencias
├── README.md            # Documentación de usuario
└── run*.sh              # Scripts de ejecución
\end{lstlisting}

\section{Capa de Modelo (Models)}

La capa de modelo contiene toda la lógica de negocio y el acceso a datos. Está compuesta por tres módulos principales.

\subsection{GeneticAlgorithmTSP}

El módulo \texttt{genetic\_algorithm.py} implementa el algoritmo genético para resolver el TSP.

\textbf{Responsabilidades:}
\begin{itemize}
    \item Inicialización de población
    \item Aplicación de operadores genéticos (selección, cruce, mutación)
    \item Evaluación de individuos
    \item Evolución generacional
    \item Manejo de estadísticas
\end{itemize}

\textbf{Clase Principal:}

\begin{lstlisting}[caption=Clase GeneticAlgorithmTSP]
class GeneticAlgorithmTSP:
    """Algoritmo Genético para resolver TSP."""
    
    def __init__(self, dist_matrix, pop_size, crossover_rate, 
                 mutation_rate, num_generations, mpi_map=None):
        # Inicialización del algoritmo
        
    def run(self):
        # Ejecuta el algoritmo y retorna resultados
        return best_route, best_distance, total_time, stats
\end{lstlisting}

\textbf{Operadores Genéticos:}
\begin{itemize}
    \item \textbf{Selección:} Tournament Selection (tamaño 3)
    \item \textbf{Cruce:} Ordered Crossover (OX)
    \item \textbf{Mutación:} Shuffle Indexes Mutation
    \item \textbf{Evaluación:} Cálculo de distancia total del recorrido
\end{itemize}

\subsection{MPIHandler}

El módulo \texttt{mpi\_handler.py} maneja toda la comunicación y distribución de tareas usando MPI.

\textbf{Responsabilidades:}
\begin{itemize}
    \item Detección de MPI
    \item Identificación de rol (maestro/esclavo)
    \item Distribución de matriz de distancias
    \item Creación de mapper personalizado para MPI
    \item Comunicación maestro-esclavo
\end{itemize}

\textbf{Flujo de Comunicación:}

\begin{figure}[H]
\centering
\vspace{0.5cm}
\begin{tikzpicture}[
    box/.style={rectangle, rounded corners, fill=azulclaro!50, draw=azuloscuro, thick, minimum width=3cm, minimum height=1.2cm, text centered, font=\small},
    arrow/.style={->, >=stealth, thick, color=azuloscuro, line width=1.5pt}
]
    \node[box, fill=azulmedio!30] (master) {Maestro};
    \node[box, fill=verde!30, below=2.5cm of master] (slave) {Esclavo};
    
    \draw[arrow] (master) -- node[right, font=\footnotesize, align=left] {Tag 100:\\Matriz} ([xshift=0.3cm]slave.north);
    \draw[arrow] ([xshift=-0.3cm]master.south) -- node[left, font=\footnotesize, align=right] {Tag 1:\\Tarea} ([xshift=-0.3cm]slave.north);
    \draw[arrow] ([xshift=-0.3cm]slave.south) -- node[left, font=\footnotesize, align=right] {Tag 2:\\Resultado} ([xshift=-0.3cm]master.south);
    \draw[arrow] (master) -- node[right, font=\footnotesize, align=left] {Tag 99:\\Terminar} ([xshift=0.3cm]slave.north);
\end{tikzpicture}
\vspace{0.3cm}
\caption{Comunicación MPI entre Maestro y Esclavo}
\label{fig:mpi_communication}
\end{figure}

\subsection{DatabaseManager}

El módulo \texttt{database.py} gestiona todas las operaciones con la base de datos MySQL.

\textbf{Responsabilidades:}
\begin{itemize}
    \item Conexión a base de datos
    \item Guardado de ejecuciones
    \item Recuperación de historial
    \item Manejo de errores
\end{itemize}

\textbf{Operaciones Principales:}
\begin{itemize}
    \item \texttt{save\_execution()}: Guarda resultados de una ejecución
    \item \texttt{get\_history()}: Obtiene historial de ejecuciones
    \item \texttt{is\_available()}: Verifica disponibilidad de BD
\end{itemize}

\section{Capa de Vista (Views)}

La capa de vista maneja toda la interfaz de usuario usando tkinter.

\subsection{MainWindow}

El módulo \texttt{gui.py} implementa la ventana principal de la aplicación.

\textbf{Componentes Principales:}

\begin{enumerate}
    \item \textbf{Panel de Configuración:}
    \begin{itemize}
        \item Número de ciudades
        \item Parámetros del algoritmo
        \item Botón de ejecución
        \item Información del cluster
    \end{itemize}
    
    \item \textbf{Panel de Resultados:}
    \begin{itemize}
        \item Gráfica de convergencia (Matplotlib)
        \item Tabla de estadísticas por generación
        \item Mejor solución encontrada
        \item Tiempo de ejecución
        \item Barra de progreso
    \end{itemize}
\end{enumerate}

\textbf{Métodos Principales:}

\begin{lstlisting}[caption=Métodos de MainWindow]
class MainWindow:
    def __init__(self, root, controller):
        # Inicialización de la interfaz
        
    def update_progress(self, generation, best, worst, avg, std_dev):
        # Actualiza la vista con el progreso
        
    def show_final_results(self, best_route, best_distance, total_time):
        # Muestra los resultados finales
\end{lstlisting}

\section{Capa de Controlador (Controllers)}

El controlador coordina la comunicación entre modelos y vistas.

\subsection{AppController}

El módulo \texttt{app\_controller.py} implementa el controlador principal.

\textbf{Responsabilidades:}
\begin{itemize}
    \item Inicialización de componentes
    \item Gestión del flujo de la aplicación
    \item Coordinación entre modelo y vista
    \item Manejo de eventos de usuario
\end{itemize}

\textbf{Flujo de Ejecución:}

\begin{figure}[H]
\centering
\vspace{0.5cm}
\begin{tikzpicture}[
    box/.style={rectangle, rounded corners, fill=azulclaro!50, draw=azuloscuro, thick, minimum width=3.2cm, minimum height=1cm, text centered, font=\small},
    arrow/.style={->, >=stealth, thick, color=azuloscuro, line width=1.5pt}
]
    \node[box, fill=azulmedio!30] (user) {Usuario};
    \node[box, fill=azulclaro!30, below=0.8cm of user] (view) {Vista};
    \node[box, fill=azulmedio!30, below=0.8cm of view] (controller) {Controlador};
    \node[box, fill=verde!30, below left=1.2cm and 1.8cm of controller] (model) {Modelo};
    \node[box, fill=verde!30, below right=1.2cm and 1.8cm of controller] (mpi) {MPI};
    
    \draw[arrow] (user) -- node[right, font=\footnotesize] {Evento} (view);
    \draw[arrow] (view) -- node[right, font=\footnotesize] {Acción} (controller);
    \draw[arrow] (controller) -- node[left, font=\footnotesize] {Ejecuta} (model);
    \draw[arrow] (controller) -- node[right, font=\footnotesize] {Distribuye} (mpi);
    \draw[arrow, dashed] (model) -- node[below, font=\footnotesize] {Resultados} (controller);
    \draw[arrow, dashed] (controller) -- node[right, font=\footnotesize] {Actualiza} (view);
    \draw[arrow, dashed] (view) -- node[right, font=\footnotesize] {Muestra} (user);
\end{tikzpicture}
\vspace{0.3cm}
\caption{Flujo de ejecución en el sistema MVC}
\label{fig:execution_flow}
\end{figure}

\textbf{Métodos Principales:}

\begin{lstlisting}[caption=Métodos de AppController]
class AppController:
    def __init__(self, view):
        # Inicialización
        
    def load_matrix(self, filepath):
        # Carga matriz de distancias
        
    def execute_algorithm(self, params):
        # Ejecuta el algoritmo genético
\end{lstlisting}

\section{Utilidades}

\subsection{MatrixLoader}

El módulo \texttt{matrix\_loader.py} proporciona utilidades para cargar y generar matrices de distancias.

\textbf{Funcionalidades:}
\begin{itemize}
    \item Carga desde archivo JSON
    \item Carga desde archivo de texto
    \item Generación de matrices aleatorias
    \item Validación de matrices
\end{itemize}

\section{Configuración}

\subsection{config.py}

El archivo \texttt{config.py} centraliza toda la configuración del sistema.

\textbf{Configuraciones Principales:}
\begin{itemize}
    \item Configuración de base de datos
    \item Configuración de MPI
    \item Rutas de archivos
    \item Valores por defecto
\end{itemize}

\section{Punto de Entrada}

\subsection{main.py}

El archivo \texttt{main.py} es el punto de entrada principal de la aplicación.

\textbf{Flujo Principal:}

\begin{enumerate}
    \item Detección de MPI y rol (maestro/esclavo)
    \item Si es esclavo: entrar en bucle de procesamiento
    \item Si es maestro: inicializar interfaz gráfica
    \item Crear vista, controlador y conectar componentes
    \item Ejecutar bucle principal de tkinter
\end{enumerate}

\begin{lstlisting}[caption=Flujo Principal en main.py]
if MPI_AVAILABLE and rank != 0:
    # ESCLAVO: Procesar tareas
    while True:
        recibir_tarea()
        procesar()
        enviar_resultado()
else:
    # MAESTRO: Ejecutar interfaz
    root = tk.Tk()
    view = MainWindow(root, None)
    controller = AppController(view)
    view.controller = controller
    root.mainloop()
\end{lstlisting}

\section{Ejecución del Sistema}

Esta sección proporciona una guía práctica paso a paso para inicializar y ejecutar el sistema.

\subsection{Requisitos Previos}

Antes de inicializar el programa, es necesario verificar que se cumplan los siguientes requisitos:

\begin{enumerate}
    \item \textbf{Python 3.8 o superior:} Verificar versión con:
    \begin{lstlisting}[caption=Verificar versión de Python, language=bash]
python3 --version
    \end{lstlisting}
    
    \item \textbf{OpenMPI (opcional, para modo distribuido):} Verificar instalación con:
    \begin{lstlisting}[caption=Verificar MPI, language=bash]
mpirun --version
    \end{lstlisting}
    
    \item \textbf{MySQL (opcional, para guardar resultados):} Solo necesario si se desea persistir los resultados en base de datos.
\end{enumerate}

\subsection{Instalación de Dependencias}

\subsubsection{Paso 1: Crear Entorno Virtual}

Se recomienda crear un entorno virtual para aislar las dependencias del proyecto:

\begin{lstlisting}[caption=Crear entorno virtual, language=bash]
# Crear el entorno virtual
python3 -m venv venv

# Activar el entorno virtual
# En Linux/Mac:
source venv/bin/activate

# En Windows:
# venv\Scripts\activate
\end{lstlisting}

\subsubsection{Paso 2: Instalar Dependencias Python}

Una vez activado el entorno virtual, instalar las dependencias desde el archivo \texttt{requirements.txt}:

\begin{lstlisting}[caption=Instalar dependencias, language=bash]
pip install -r requirements.txt
\end{lstlisting}

Las dependencias principales incluyen:
\begin{itemize}
    \item \texttt{numpy}: Operaciones matemáticas y matrices
    \item \texttt{matplotlib}: Visualización de gráficas
    \item \texttt{deap}: Framework de algoritmos evolutivos
    \item \texttt{mpi4py}: Interfaz Python para MPI
    \item \texttt{mysql-connector-python}: Conector para MySQL (opcional)
\end{itemize}

\subsubsection{Paso 3: Verificar Instalación}

Verificar que todas las dependencias se instalaron correctamente:

\begin{lstlisting}[caption=Verificar instalación, language=bash]
python -c "import numpy, matplotlib, deap, mpi4py; print('Dependencias OK')"
\end{lstlisting}

\subsection{Configuración Inicial}

\subsubsection{Configuración de Base de Datos (Opcional)}

Si se desea usar la funcionalidad de base de datos, editar el archivo \texttt{config/config.py}:

\begin{lstlisting}[caption=Configuración de BD en config/config.py]
DB_CONFIG = {
    "host": "localhost",
    "user": "aguser",
    "password": "password123",
    "database": "base_de_datos_replicacion",
    "port": 3306
}
\end{lstlisting}

También se pueden configurar mediante variables de entorno:
\begin{lstlisting}[caption=Variables de entorno para BD, language=bash]
export DB_HOST=localhost
export DB_USER=aguser
export DB_PASSWORD=password123
export DB_NAME=base_de_datos_replicacion
export DB_PORT=3306
\end{lstlisting}

\subsubsection{Configuración de MPI (Opcional)}

Para ejecución distribuida, crear un archivo \texttt{mpi\_hostfile} con los nodos del cluster:

\begin{lstlisting}[caption=Ejemplo de mpi_hostfile, language=bash]
# Formato: hostname slots=N
nodo1 slots=2
nodo2 slots=2
nodo3 slots=2
\end{lstlisting}

\subsection{Inicialización del Programa}

\subsubsection{Modo Local (Sin MPI)}

Para ejecutar el programa en modo local sin computación distribuida:

\textbf{Método 1: Ejecución directa con Python}

\begin{lstlisting}[caption=Ejecución Local Directa, language=bash]
# Asegurarse de estar en el directorio del proyecto
cd /ruta/al/proyecto/bueno

# Activar entorno virtual (si se usa)
source venv/bin/activate

# Ejecutar el programa
python main.py
\end{lstlisting}

\textbf{Método 2: Usando el script de ejecución}

\begin{lstlisting}[caption=Ejecución con script, language=bash]
# Dar permisos de ejecución (solo la primera vez)
chmod +x run.sh

# Ejecutar el script
./run.sh
\end{lstlisting}

El script \texttt{run.sh} automáticamente:
\begin{itemize}
    \item Cambia al directorio del proyecto
    \item Activa el entorno virtual si existe
    \item Ejecuta \texttt{main.py}
\end{itemize}

\subsubsection{Modo Distribuido (Con MPI)}

Para ejecutar el programa en modo distribuido con múltiples procesos:

\textbf{Método 1: Ejecución local con múltiples procesos}

\begin{lstlisting}[caption=MPI Local con 4 procesos, language=bash]
# Ejecutar con 4 procesos (1 maestro + 3 esclavos)
mpirun -np 4 python main.py
\end{lstlisting}

\textbf{Método 2: Ejecución en cluster con hostfile}

\begin{lstlisting}[caption=MPI con hostfile, language=bash]
# Ejecutar usando nodos definidos en hostfile
mpirun --hostfile mpi_hostfile -np 8 python main.py
\end{lstlisting}

\textbf{Método 3: Usando el script de ejecución MPI}

\begin{lstlisting}[caption=Ejecución con script MPI, language=bash]
# Dar permisos de ejecución (solo la primera vez)
chmod +x run_mpi.sh

# Ejecutar con número de procesos por defecto (4)
./run_mpi.sh

# Ejecutar con número de procesos personalizado
./run_mpi.sh 8
\end{lstlisting}

\subsection{Flujo de Inicialización}

El programa sigue el siguiente flujo al inicializarse:

\begin{enumerate}
    \item \textbf{Detección de MPI:} El programa intenta importar \texttt{mpi4py} y detecta si está disponible.
    
    \item \textbf{Identificación de Rol:}
    \begin{itemize}
        \item Si MPI está disponible y el rank es 0: se ejecuta como \textbf{maestro}
        \item Si MPI está disponible y el rank es diferente de 0: se ejecuta como \textbf{esclavo}
        \item Si MPI no está disponible: se ejecuta en \textbf{modo local}
    \end{itemize}
    
    \item \textbf{Proceso Esclavo (si aplica):}
    \begin{itemize}
        \item Entra en un bucle de espera de mensajes
        \item Recibe la matriz de distancias del maestro
        \item Procesa tareas de evaluación de individuos
        \item Envía resultados de vuelta al maestro
        \item Termina cuando recibe señal de terminación
    \end{itemize}
    
    \item \textbf{Proceso Maestro:}
    \begin{itemize}
        \item Inicializa la interfaz gráfica (tkinter)
        \item Crea la vista (\texttt{MainWindow})
        \item Crea el controlador (\texttt{AppController})
        \item Conecta la vista con el controlador
        \item Inicia el bucle principal de la interfaz gráfica
    \end{itemize}
\end{enumerate}

\subsection{Verificación de Inicialización Correcta}

Al iniciar el programa, se deben ver mensajes en la consola indicando el estado:

\textbf{Modo Local:}
\begin{lstlisting}[caption=Salida esperada en modo local]
[MAESTRO] Iniciando aplicación en modo local...
[INFO] Matriz cargada: 17 ciudades
\end{lstlisting}

\textbf{Modo Distribuido:}
\begin{lstlisting}[caption=Salida esperada en modo distribuido]
[MAESTRO Rank 0] Iniciando aplicación con 3 esclavos...
[ESCLAVO Rank 1] Iniciando proceso esclavo...
[ESCLAVO Rank 2] Iniciando proceso esclavo...
[ESCLAVO Rank 3] Iniciando proceso esclavo...
[ESCLAVO Rank 1] Esperando mensajes del maestro...
[ESCLAVO Rank 2] Esperando mensajes del maestro...
[ESCLAVO Rank 3] Esperando mensajes del maestro...
[INFO] Matriz cargada: 17 ciudades
\end{lstlisting}

\subsection{Solución de Problemas Comunes}

\begin{itemize}
    \item \textbf{Error: "No module named 'mpi4py'":} 
    \begin{itemize}
        \item Instalar MPI: \texttt{sudo apt-get install openmpi-bin libopenmpi-dev} (Linux)
        \item Instalar mpi4py: \texttt{pip install mpi4py}
    \end{itemize}
    
    \item \textbf{Error: "No module named 'tkinter'":}
    \begin{itemize}
        \item Instalar tkinter: \texttt{sudo apt-get install python3-tk} (Linux)
    \end{itemize}
    
    \item \textbf{Error: "Cannot connect to database":}
    \begin{itemize}
        \item Verificar configuración en \texttt{config/config.py}
        \item Verificar que MySQL esté ejecutándose
        \item El programa funcionará sin base de datos, solo no guardará resultados
    \end{itemize}
    
    \item \textbf{Error: "Permission denied" en scripts:}
    \begin{itemize}
        \item Dar permisos: \texttt{chmod +x run.sh run\_mpi.sh}
    \end{itemize}
\end{itemize}

\section{Ventajas de la Arquitectura MVC}

\begin{enumerate}
    \item \textbf{Separación de Responsabilidades:} Cada componente tiene una función clara
    \item \textbf{Modularidad:} Fácil de mantener y extender
    \item \textbf{Testabilidad:} Cada componente se puede probar independientemente
    \item \textbf{Reutilización:} Los modelos pueden usarse en diferentes vistas
    \item \textbf{Escalabilidad:} Fácil agregar nuevas funcionalidades
\end{enumerate}

\section{Conclusión}

Este sistema implementa una arquitectura MVC limpia y bien estructurada para resolver el problema TSP usando algoritmos genéticos distribuidos. La separación clara de responsabilidades facilita el mantenimiento, la extensión y la comprensión del código.

\newpage

\section{Referencias}

\begin{enumerate}
    \item DEAP - Distributed Evolutionary Algorithms in Python
    \item MPI for Python (mpi4py) Documentation
    \item Pattern: Model-View-Controller (MVC)
    \item Tkinter Documentation
    \item MySQL Connector/Python Documentation
\end{enumerate}

\end{document}




